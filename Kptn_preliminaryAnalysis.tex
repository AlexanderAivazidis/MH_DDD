\documentclass[]{article}
\usepackage{lmodern}
\usepackage{amssymb,amsmath}
\usepackage{ifxetex,ifluatex}
\usepackage{fixltx2e} % provides \textsubscript
\ifnum 0\ifxetex 1\fi\ifluatex 1\fi=0 % if pdftex
  \usepackage[T1]{fontenc}
  \usepackage[utf8]{inputenc}
\else % if luatex or xelatex
  \ifxetex
    \usepackage{mathspec}
  \else
    \usepackage{fontspec}
  \fi
  \defaultfontfeatures{Ligatures=TeX,Scale=MatchLowercase}
\fi
% use upquote if available, for straight quotes in verbatim environments
\IfFileExists{upquote.sty}{\usepackage{upquote}}{}
% use microtype if available
\IfFileExists{microtype.sty}{%
\usepackage{microtype}
\UseMicrotypeSet[protrusion]{basicmath} % disable protrusion for tt fonts
}{}
\usepackage[margin=1in]{geometry}
\usepackage{hyperref}
\hypersetup{unicode=true,
            pdfborder={0 0 0},
            breaklinks=true}
\urlstyle{same}  % don't use monospace font for urls
\usepackage{color}
\usepackage{fancyvrb}
\newcommand{\VerbBar}{|}
\newcommand{\VERB}{\Verb[commandchars=\\\{\}]}
\DefineVerbatimEnvironment{Highlighting}{Verbatim}{commandchars=\\\{\}}
% Add ',fontsize=\small' for more characters per line
\usepackage{framed}
\definecolor{shadecolor}{RGB}{248,248,248}
\newenvironment{Shaded}{\begin{snugshade}}{\end{snugshade}}
\newcommand{\AlertTok}[1]{\textcolor[rgb]{0.94,0.16,0.16}{#1}}
\newcommand{\AnnotationTok}[1]{\textcolor[rgb]{0.56,0.35,0.01}{\textbf{\textit{#1}}}}
\newcommand{\AttributeTok}[1]{\textcolor[rgb]{0.77,0.63,0.00}{#1}}
\newcommand{\BaseNTok}[1]{\textcolor[rgb]{0.00,0.00,0.81}{#1}}
\newcommand{\BuiltInTok}[1]{#1}
\newcommand{\CharTok}[1]{\textcolor[rgb]{0.31,0.60,0.02}{#1}}
\newcommand{\CommentTok}[1]{\textcolor[rgb]{0.56,0.35,0.01}{\textit{#1}}}
\newcommand{\CommentVarTok}[1]{\textcolor[rgb]{0.56,0.35,0.01}{\textbf{\textit{#1}}}}
\newcommand{\ConstantTok}[1]{\textcolor[rgb]{0.00,0.00,0.00}{#1}}
\newcommand{\ControlFlowTok}[1]{\textcolor[rgb]{0.13,0.29,0.53}{\textbf{#1}}}
\newcommand{\DataTypeTok}[1]{\textcolor[rgb]{0.13,0.29,0.53}{#1}}
\newcommand{\DecValTok}[1]{\textcolor[rgb]{0.00,0.00,0.81}{#1}}
\newcommand{\DocumentationTok}[1]{\textcolor[rgb]{0.56,0.35,0.01}{\textbf{\textit{#1}}}}
\newcommand{\ErrorTok}[1]{\textcolor[rgb]{0.64,0.00,0.00}{\textbf{#1}}}
\newcommand{\ExtensionTok}[1]{#1}
\newcommand{\FloatTok}[1]{\textcolor[rgb]{0.00,0.00,0.81}{#1}}
\newcommand{\FunctionTok}[1]{\textcolor[rgb]{0.00,0.00,0.00}{#1}}
\newcommand{\ImportTok}[1]{#1}
\newcommand{\InformationTok}[1]{\textcolor[rgb]{0.56,0.35,0.01}{\textbf{\textit{#1}}}}
\newcommand{\KeywordTok}[1]{\textcolor[rgb]{0.13,0.29,0.53}{\textbf{#1}}}
\newcommand{\NormalTok}[1]{#1}
\newcommand{\OperatorTok}[1]{\textcolor[rgb]{0.81,0.36,0.00}{\textbf{#1}}}
\newcommand{\OtherTok}[1]{\textcolor[rgb]{0.56,0.35,0.01}{#1}}
\newcommand{\PreprocessorTok}[1]{\textcolor[rgb]{0.56,0.35,0.01}{\textit{#1}}}
\newcommand{\RegionMarkerTok}[1]{#1}
\newcommand{\SpecialCharTok}[1]{\textcolor[rgb]{0.00,0.00,0.00}{#1}}
\newcommand{\SpecialStringTok}[1]{\textcolor[rgb]{0.31,0.60,0.02}{#1}}
\newcommand{\StringTok}[1]{\textcolor[rgb]{0.31,0.60,0.02}{#1}}
\newcommand{\VariableTok}[1]{\textcolor[rgb]{0.00,0.00,0.00}{#1}}
\newcommand{\VerbatimStringTok}[1]{\textcolor[rgb]{0.31,0.60,0.02}{#1}}
\newcommand{\WarningTok}[1]{\textcolor[rgb]{0.56,0.35,0.01}{\textbf{\textit{#1}}}}
\usepackage{graphicx,grffile}
\makeatletter
\def\maxwidth{\ifdim\Gin@nat@width>\linewidth\linewidth\else\Gin@nat@width\fi}
\def\maxheight{\ifdim\Gin@nat@height>\textheight\textheight\else\Gin@nat@height\fi}
\makeatother
% Scale images if necessary, so that they will not overflow the page
% margins by default, and it is still possible to overwrite the defaults
% using explicit options in \includegraphics[width, height, ...]{}
\setkeys{Gin}{width=\maxwidth,height=\maxheight,keepaspectratio}
\IfFileExists{parskip.sty}{%
\usepackage{parskip}
}{% else
\setlength{\parindent}{0pt}
\setlength{\parskip}{6pt plus 2pt minus 1pt}
}
\setlength{\emergencystretch}{3em}  % prevent overfull lines
\providecommand{\tightlist}{%
  \setlength{\itemsep}{0pt}\setlength{\parskip}{0pt}}
\setcounter{secnumdepth}{0}
% Redefines (sub)paragraphs to behave more like sections
\ifx\paragraph\undefined\else
\let\oldparagraph\paragraph
\renewcommand{\paragraph}[1]{\oldparagraph{#1}\mbox{}}
\fi
\ifx\subparagraph\undefined\else
\let\oldsubparagraph\subparagraph
\renewcommand{\subparagraph}[1]{\oldsubparagraph{#1}\mbox{}}
\fi

%%% Use protect on footnotes to avoid problems with footnotes in titles
\let\rmarkdownfootnote\footnote%
\def\footnote{\protect\rmarkdownfootnote}

%%% Change title format to be more compact
\usepackage{titling}

% Create subtitle command for use in maketitle
\providecommand{\subtitle}[1]{
  \posttitle{
    \begin{center}\large#1\end{center}
    }
}

\setlength{\droptitle}{-2em}

  \title{}
    \pretitle{\vspace{\droptitle}}
  \posttitle{}
    \author{}
    \preauthor{}\postauthor{}
    \date{}
    \predate{}\postdate{}
  

\begin{document}

\hypertarget{test-preliminary-analysis}{%
\subsubsection{Test Preliminary
analysis}\label{test-preliminary-analysis}}

\begin{Shaded}
\begin{Highlighting}[]
\KeywordTok{setwd}\NormalTok{(}\StringTok{'/home/jovyan/MH_DDD/'}\NormalTok{)}

\NormalTok{tab =}\StringTok{ }\KeywordTok{read.delim}\NormalTok{(}\StringTok{'../data/KptnMouse/RNAscope/old/Objects_Population - Nuclei.txt'}\NormalTok{, }\DataTypeTok{sep =} \StringTok{'}\CharTok{\textbackslash{}t}\StringTok{'}\NormalTok{, }\DataTypeTok{skip =} \DecValTok{9}\NormalTok{)}
\NormalTok{tab =}\StringTok{ }\NormalTok{tab[}\KeywordTok{order}\NormalTok{(}\OperatorTok{-}\NormalTok{tab}\OperatorTok{$}\NormalTok{Position.Y..µm.),]}
\end{Highlighting}
\end{Shaded}

Remove too large (doublets) and too small (partial) nuclei:

\begin{Shaded}
\begin{Highlighting}[]
\KeywordTok{hist}\NormalTok{(tab}\OperatorTok{$}\NormalTok{Nuclei...Nucleus.Volume..µm.., }\DataTypeTok{breaks =} \DecValTok{100}\NormalTok{, }\DataTypeTok{main =} \StringTok{'Nuclei Volume histogram before filtering'}\NormalTok{)}
\end{Highlighting}
\end{Shaded}

\includegraphics{Kptn_preliminaryAnalysis_files/figure-latex/unnamed-chunk-4-1.pdf}

\begin{Shaded}
\begin{Highlighting}[]
\KeywordTok{plot}\NormalTok{(tab}\OperatorTok{$}\NormalTok{Position.Y..µm., tab}\OperatorTok{$}\NormalTok{Position.X..µm., }\DataTypeTok{pch =} \StringTok{'.'}\NormalTok{, }\DataTypeTok{main =} \StringTok{'Nuclei position before filtering'}\NormalTok{)}
\end{Highlighting}
\end{Shaded}

\includegraphics{Kptn_preliminaryAnalysis_files/figure-latex/unnamed-chunk-4-2.pdf}

\begin{Shaded}
\begin{Highlighting}[]
\NormalTok{keep =}\StringTok{ }\NormalTok{tab}\OperatorTok{$}\NormalTok{Nuclei...Nucleus.Volume..µm.. }\OperatorTok{<}\StringTok{ }\DecValTok{4000}
\KeywordTok{print}\NormalTok{(}\KeywordTok{sum}\NormalTok{(keep)}\OperatorTok{/}\KeywordTok{dim}\NormalTok{(tab)[}\DecValTok{1}\NormalTok{])}
\end{Highlighting}
\end{Shaded}

\begin{verbatim}
## [1] 0.9448368
\end{verbatim}

\begin{Shaded}
\begin{Highlighting}[]
\NormalTok{tab =}\StringTok{ }\NormalTok{tab[keep,]}
\NormalTok{keep =}\StringTok{ }\NormalTok{tab}\OperatorTok{$}\NormalTok{Nuclei...Nucleus.Volume..µm.. }\OperatorTok{>}\StringTok{ }\DecValTok{1000}
\KeywordTok{print}\NormalTok{(}\KeywordTok{sum}\NormalTok{(keep)}\OperatorTok{/}\KeywordTok{dim}\NormalTok{(tab)[}\DecValTok{1}\NormalTok{])}
\end{Highlighting}
\end{Shaded}

\begin{verbatim}
## [1] 0.993829
\end{verbatim}

\begin{Shaded}
\begin{Highlighting}[]
\NormalTok{tab =}\StringTok{ }\NormalTok{tab[keep,]}
\KeywordTok{hist}\NormalTok{(tab}\OperatorTok{$}\NormalTok{Nuclei...Nucleus.Volume..µm.., }\DataTypeTok{breaks =} \DecValTok{100}\NormalTok{, }\DataTypeTok{main =} \StringTok{'Nuclei Volume histogram after filtering'}\NormalTok{)}
\end{Highlighting}
\end{Shaded}

\includegraphics{Kptn_preliminaryAnalysis_files/figure-latex/unnamed-chunk-4-3.pdf}

\begin{Shaded}
\begin{Highlighting}[]
\KeywordTok{plot}\NormalTok{(tab}\OperatorTok{$}\NormalTok{Position.Y..µm., tab}\OperatorTok{$}\NormalTok{Position.X..µm., }\DataTypeTok{pch =} \StringTok{'.'}\NormalTok{, }\DataTypeTok{main =} \StringTok{'Nuclei position after filtering'}\NormalTok{)}
\end{Highlighting}
\end{Shaded}

\includegraphics{Kptn_preliminaryAnalysis_files/figure-latex/unnamed-chunk-4-4.pdf}

Make 1D histogram plots of intensity (sum, mean or normalized sum)
within nuclei for the first section only:

\begin{Shaded}
\begin{Highlighting}[]
\NormalTok{sectionNumber =}\StringTok{ }\KeywordTok{rep}\NormalTok{(}\DecValTok{0}\NormalTok{, }\KeywordTok{dim}\NormalTok{(tab)[}\DecValTok{1}\NormalTok{])}
\NormalTok{count =}\StringTok{ }\DecValTok{1}
\NormalTok{sectionNumber[}\DecValTok{1}\NormalTok{] =}\StringTok{ }\NormalTok{count}
\ControlFlowTok{for}\NormalTok{ (i }\ControlFlowTok{in} \DecValTok{2}\OperatorTok{:}\KeywordTok{dim}\NormalTok{(tab))\{}
\ControlFlowTok{if}\NormalTok{ (}\KeywordTok{abs}\NormalTok{(tab}\OperatorTok{$}\NormalTok{Position.Y..µm.[i] }\OperatorTok{-}\StringTok{ }\NormalTok{tab}\OperatorTok{$}\NormalTok{Position.Y..µm.[i}\DecValTok{-1}\NormalTok{]) }\OperatorTok{>}\StringTok{ }\DecValTok{1000}\NormalTok{)\{count =}\StringTok{ }\NormalTok{count }\OperatorTok{+}\StringTok{ }\DecValTok{1}\NormalTok{\}}
\NormalTok{sectionNumber[i] =}\StringTok{ }\NormalTok{count}
\NormalTok{\}}

\NormalTok{section =}\StringTok{ }\DecValTok{1}

\NormalTok{meanIntensity =}\StringTok{ }\KeywordTok{as.matrix}\NormalTok{(tab[sectionNumber }\OperatorTok{==}\StringTok{ }\NormalTok{section,}\KeywordTok{c}\NormalTok{(}\StringTok{'Nuclei...Intensity.Nucleus.Atto.490LS.Mean'}\NormalTok{, }\StringTok{'Nuclei...Intensity.Nucleus.Atto.425.Mean'}\NormalTok{,}
                         \StringTok{'Nuclei...Intensity.Nucleus.Alexa.488.Mean'}\NormalTok{, }\StringTok{'Nuclei...Intensity.Nucleus.Alexa.568.Mean'}\NormalTok{,}
                         \StringTok{'Nuclei...Intensity.Nucleus.Alexa.647.Mean'}\NormalTok{)])}
\NormalTok{sumIntensity =}\StringTok{ }\KeywordTok{as.matrix}\NormalTok{(tab[sectionNumber }\OperatorTok{==}\StringTok{ }\NormalTok{section,}\KeywordTok{c}\NormalTok{(}\StringTok{'Nuclei...Intensity.Nucleus.Atto.490LS.Sum'}\NormalTok{, }\StringTok{'Nuclei...Intensity.Nucleus.Atto.425.Sum'}\NormalTok{,}
                         \StringTok{'Nuclei...Intensity.Nucleus.Alexa.488.Sum'}\NormalTok{, }\StringTok{'Nuclei...Intensity.Nucleus.Alexa.568.Sum'}\NormalTok{,}
                         \StringTok{'Nuclei...Intensity.Nucleus.Alexa.647.Sum'}\NormalTok{)])}
\NormalTok{normSumIntensity =}\StringTok{ }\NormalTok{sumIntensity}\OperatorTok{/}\KeywordTok{rowSums}\NormalTok{(sumIntensity)}

\NormalTok{channels =}\StringTok{ }\KeywordTok{c}\NormalTok{(}\StringTok{'490LS'}\NormalTok{, }\StringTok{'425'}\NormalTok{, }\StringTok{'488'}\NormalTok{, }\StringTok{'568'}\NormalTok{, }\StringTok{'647'}\NormalTok{)}
\ControlFlowTok{for}\NormalTok{ (i }\ControlFlowTok{in} \DecValTok{1}\OperatorTok{:}\KeywordTok{length}\NormalTok{(channels))\{}
\KeywordTok{par}\NormalTok{(}\DataTypeTok{mfrow =} \KeywordTok{c}\NormalTok{(}\DecValTok{1}\NormalTok{,}\DecValTok{3}\NormalTok{))}
\KeywordTok{hist}\NormalTok{(}\KeywordTok{log}\NormalTok{(meanIntensity[, i], }\DecValTok{2}\NormalTok{), }\DecValTok{100}\NormalTok{, }\DataTypeTok{main =} \KeywordTok{paste}\NormalTok{(}\StringTok{'Histogram of'}\NormalTok{, channels[i], }\StringTok{'Mean Log-Intensity'}\NormalTok{))}
\KeywordTok{hist}\NormalTok{(}\KeywordTok{log}\NormalTok{(sumIntensity[, i], }\DecValTok{2}\NormalTok{), }\DecValTok{100}\NormalTok{, }\DataTypeTok{main =} \KeywordTok{paste}\NormalTok{(}\StringTok{'Histogram of'}\NormalTok{, channels[i], }\StringTok{'Log-Intensity Sum'}\NormalTok{))}
\KeywordTok{hist}\NormalTok{(normSumIntensity[, i], }\DecValTok{100}\NormalTok{, }\DataTypeTok{main =} \KeywordTok{paste}\NormalTok{(}\StringTok{'Histogram of'}\NormalTok{, channels[i], }\StringTok{'Normalized Intensity Sum'}\NormalTok{))}
\NormalTok{\}}
\end{Highlighting}
\end{Shaded}

\includegraphics{Kptn_preliminaryAnalysis_files/figure-latex/unnamed-chunk-5-1.pdf}
\includegraphics{Kptn_preliminaryAnalysis_files/figure-latex/unnamed-chunk-5-2.pdf}
\includegraphics{Kptn_preliminaryAnalysis_files/figure-latex/unnamed-chunk-5-3.pdf}
\includegraphics{Kptn_preliminaryAnalysis_files/figure-latex/unnamed-chunk-5-4.pdf}
\includegraphics{Kptn_preliminaryAnalysis_files/figure-latex/unnamed-chunk-5-5.pdf}

\begin{Shaded}
\begin{Highlighting}[]
\NormalTok{sectionNumber =}\StringTok{ }\KeywordTok{rep}\NormalTok{(}\DecValTok{0}\NormalTok{, }\KeywordTok{dim}\NormalTok{(tab)[}\DecValTok{1}\NormalTok{])}
\NormalTok{count =}\StringTok{ }\DecValTok{1}
\NormalTok{sectionNumber[}\DecValTok{1}\NormalTok{] =}\StringTok{ }\NormalTok{count}
\ControlFlowTok{for}\NormalTok{ (i }\ControlFlowTok{in} \DecValTok{2}\OperatorTok{:}\KeywordTok{dim}\NormalTok{(tab))\{}
\ControlFlowTok{if}\NormalTok{ (}\KeywordTok{abs}\NormalTok{(tab}\OperatorTok{$}\NormalTok{Position.Y..µm.[i] }\OperatorTok{-}\StringTok{ }\NormalTok{tab}\OperatorTok{$}\NormalTok{Position.Y..µm.[i}\DecValTok{-1}\NormalTok{]) }\OperatorTok{>}\StringTok{ }\DecValTok{1000}\NormalTok{)\{count =}\StringTok{ }\NormalTok{count }\OperatorTok{+}\StringTok{ }\DecValTok{1}\NormalTok{\}}
\NormalTok{sectionNumber[i] =}\StringTok{ }\NormalTok{count}
\NormalTok{\}}

\NormalTok{section =}\StringTok{ }\DecValTok{1}

\NormalTok{meanIntensity =}\StringTok{ }\KeywordTok{as.matrix}\NormalTok{(tab[sectionNumber }\OperatorTok{==}\StringTok{ }\NormalTok{section,}\KeywordTok{c}\NormalTok{(}\StringTok{'Nuclei...Intensity.Nucleus.Atto.490LS.Mean'}\NormalTok{, }\StringTok{'Nuclei...Intensity.Nucleus.Atto.425.Mean'}\NormalTok{,}
                         \StringTok{'Nuclei...Intensity.Nucleus.Alexa.488.Mean'}\NormalTok{, }\StringTok{'Nuclei...Intensity.Nucleus.Alexa.568.Mean'}\NormalTok{,}
                         \StringTok{'Nuclei...Intensity.Nucleus.Alexa.647.Mean'}\NormalTok{)])}
\NormalTok{sumIntensity =}\StringTok{ }\KeywordTok{as.matrix}\NormalTok{(tab[sectionNumber }\OperatorTok{==}\StringTok{ }\NormalTok{section,}\KeywordTok{c}\NormalTok{(}\StringTok{'Nuclei...Intensity.Nucleus.Atto.490LS.Sum'}\NormalTok{, }\StringTok{'Nuclei...Intensity.Nucleus.Atto.425.Sum'}\NormalTok{,}
                         \StringTok{'Nuclei...Intensity.Nucleus.Alexa.488.Sum'}\NormalTok{, }\StringTok{'Nuclei...Intensity.Nucleus.Alexa.568.Sum'}\NormalTok{,}
                         \StringTok{'Nuclei...Intensity.Nucleus.Alexa.647.Sum'}\NormalTok{)])}
\NormalTok{normSumIntensity =}\StringTok{ }\NormalTok{sumIntensity}\OperatorTok{/}\KeywordTok{rowSums}\NormalTok{(sumIntensity)}

\NormalTok{channels =}\StringTok{ }\KeywordTok{c}\NormalTok{(}\StringTok{'490LS'}\NormalTok{, }\StringTok{'425'}\NormalTok{, }\StringTok{'488'}\NormalTok{, }\StringTok{'568'}\NormalTok{, }\StringTok{'647'}\NormalTok{)}
\ControlFlowTok{for}\NormalTok{ (i }\ControlFlowTok{in} \DecValTok{1}\OperatorTok{:}\KeywordTok{length}\NormalTok{(channels))\{}
\KeywordTok{par}\NormalTok{(}\DataTypeTok{mfrow =} \KeywordTok{c}\NormalTok{(}\DecValTok{1}\NormalTok{,}\DecValTok{3}\NormalTok{))}
\KeywordTok{hist}\NormalTok{(}\KeywordTok{log}\NormalTok{(meanIntensity[, i], }\DecValTok{2}\NormalTok{), }\DecValTok{100}\NormalTok{, }\DataTypeTok{main =} \KeywordTok{paste}\NormalTok{(}\StringTok{'Histogram of'}\NormalTok{, channels[i], }\StringTok{'Mean Log-Intensity'}\NormalTok{))}
\KeywordTok{hist}\NormalTok{(}\KeywordTok{log}\NormalTok{(sumIntensity[, i], }\DecValTok{2}\NormalTok{), }\DecValTok{100}\NormalTok{, }\DataTypeTok{main =} \KeywordTok{paste}\NormalTok{(}\StringTok{'Histogram of'}\NormalTok{, channels[i], }\StringTok{'Log-Intensity Sum'}\NormalTok{))}
\KeywordTok{hist}\NormalTok{(normSumIntensity[, i], }\DecValTok{100}\NormalTok{, }\DataTypeTok{main =} \KeywordTok{paste}\NormalTok{(}\StringTok{'Histogram of'}\NormalTok{, channels[i], }\StringTok{'Normalized Intensity Sum'}\NormalTok{))}
\NormalTok{\}}
\end{Highlighting}
\end{Shaded}

\includegraphics{Kptn_preliminaryAnalysis_files/figure-latex/unnamed-chunk-6-1.pdf}
\includegraphics{Kptn_preliminaryAnalysis_files/figure-latex/unnamed-chunk-6-2.pdf}
\includegraphics{Kptn_preliminaryAnalysis_files/figure-latex/unnamed-chunk-6-3.pdf}
\includegraphics{Kptn_preliminaryAnalysis_files/figure-latex/unnamed-chunk-6-4.pdf}
\includegraphics{Kptn_preliminaryAnalysis_files/figure-latex/unnamed-chunk-6-5.pdf}

Summarize into the most informative features:

\begin{Shaded}
\begin{Highlighting}[]
\NormalTok{informativeFeatures =}\StringTok{ }\KeywordTok{as.matrix}\NormalTok{(}\KeywordTok{cbind}\NormalTok{(meanIntensity[,}\DecValTok{1}\OperatorTok{:}\DecValTok{4}\NormalTok{], sumIntensity[,}\DecValTok{5}\NormalTok{]))}
\end{Highlighting}
\end{Shaded}

Now make 2D histograms for each combination of channels:

\begin{Shaded}
\begin{Highlighting}[]
\KeywordTok{par}\NormalTok{(}\DataTypeTok{mfrow =} \KeywordTok{c}\NormalTok{(}\DecValTok{5}\NormalTok{,}\DecValTok{5}\NormalTok{), }\DataTypeTok{mar=}\KeywordTok{c}\NormalTok{(}\DecValTok{2}\NormalTok{,}\DecValTok{2}\NormalTok{,}\DecValTok{2}\NormalTok{,}\DecValTok{2}\NormalTok{))}
\ControlFlowTok{for}\NormalTok{ (i }\ControlFlowTok{in} \DecValTok{1}\OperatorTok{:}\KeywordTok{dim}\NormalTok{(meanIntensity)[}\DecValTok{2}\NormalTok{])\{}
  \ControlFlowTok{for}\NormalTok{ (j }\ControlFlowTok{in} \DecValTok{1}\OperatorTok{:}\KeywordTok{dim}\NormalTok{(meanIntensity)[}\DecValTok{2}\NormalTok{])\{}
    \KeywordTok{hist2d}\NormalTok{(}\KeywordTok{log}\NormalTok{(informativeFeatures[, }\KeywordTok{c}\NormalTok{(i,j)],}\DecValTok{2}\NormalTok{), }\DataTypeTok{xlab =}\NormalTok{ channels[i], }\DataTypeTok{ylab =}\NormalTok{ channels[j], }\DataTypeTok{bins =} \DecValTok{50}\NormalTok{,}
           \DataTypeTok{main =} \KeywordTok{paste}\NormalTok{(channels[i], }\StringTok{'vs.'}\NormalTok{, channels[j]))}
\NormalTok{  \}}
\NormalTok{\}}
\end{Highlighting}
\end{Shaded}

\includegraphics{Kptn_preliminaryAnalysis_files/figure-latex/unnamed-chunk-8-1.pdf}

See to what extent we can see some basic clusters in UMAP and
hierarchical clustering:

\begin{Shaded}
\begin{Highlighting}[]
\CommentTok{# intensityTab = CreateSeuratObject(log(t(meanIntensity),2))}
\CommentTok{# }
\CommentTok{# intensityTab = ScaleData(intensityTab)}
\CommentTok{# intensityTab = RunPCA(intensityTab, features = c('Nuclei...Intensity.Nucleus.Atto.490LS.Mean', 'Nuclei...Intensity.Nucleus.Atto.425.Mean',}
\CommentTok{#                          'Nuclei...Intensity.Nucleus.Alexa.488.Mean', 'Nuclei...Intensity.Nucleus.Alexa.568.Mean',}
\CommentTok{#                          'Nuclei...Intensity.Nucleus.Alexa.647.Mean'))}
\CommentTok{# intensityTab = RunUMAP(intensityTab, dims = 1:4)}
\CommentTok{# UMAPPlot(intensityTab)}
\CommentTok{# PCAPlot(intensityTab)}
\CommentTok{# featurePlots = list()}
\CommentTok{# intensityFeatures = c('Nuclei...Intensity.Nucleus.Atto.490LS.Mean', 'Nuclei...Intensity.Nucleus.Atto.425.Mean',}
\CommentTok{#                          'Nuclei...Intensity.Nucleus.Alexa.488.Mean', 'Nuclei...Intensity.Nucleus.Alexa.568.Mean',}
\CommentTok{#                          'Nuclei...Intensity.Nucleus.Alexa.647.Mean')}
\CommentTok{# for (i in 1:length(intensityFeatures))\{}
\CommentTok{#   print(i)}
\CommentTok{#   featurePlots[[i]] = FeaturePlot(intensityTab, features = intensityFeatures[[i]])}
\CommentTok{# \}}
\CommentTok{# p = cowplot::plot_grid(featurePlots[[1]], featurePlots[[2]], featurePlots[[3]], featurePlots[[4]], }
\CommentTok{#                        featurePlots[[5]])}


\KeywordTok{Heatmap}\NormalTok{(meanIntensity)}
\end{Highlighting}
\end{Shaded}

\includegraphics{Kptn_preliminaryAnalysis_files/figure-latex/unnamed-chunk-9-1.pdf}


\end{document}
